% Options for packages loaded elsewhere
\PassOptionsToPackage{unicode}{hyperref}
\PassOptionsToPackage{hyphens}{url}
\PassOptionsToPackage{dvipsnames,svgnames,x11names}{xcolor}
%
\documentclass[
]{article}
\usepackage{amsmath,amssymb}
\usepackage{iftex}
\ifPDFTeX
  \usepackage[T1]{fontenc}
  \usepackage[utf8]{inputenc}
  \usepackage{textcomp} % provide euro and other symbols
\else % if luatex or xetex
  \usepackage{unicode-math} % this also loads fontspec
  \defaultfontfeatures{Scale=MatchLowercase}
  \defaultfontfeatures[\rmfamily]{Ligatures=TeX,Scale=1}
\fi
\usepackage{lmodern}
\ifPDFTeX\else
  % xetex/luatex font selection
\fi
% Use upquote if available, for straight quotes in verbatim environments
\IfFileExists{upquote.sty}{\usepackage{upquote}}{}
\IfFileExists{microtype.sty}{% use microtype if available
  \usepackage[]{microtype}
  \UseMicrotypeSet[protrusion]{basicmath} % disable protrusion for tt fonts
}{}
\makeatletter
\@ifundefined{KOMAClassName}{% if non-KOMA class
  \IfFileExists{parskip.sty}{%
    \usepackage{parskip}
  }{% else
    \setlength{\parindent}{0pt}
    \setlength{\parskip}{6pt plus 2pt minus 1pt}}
}{% if KOMA class
  \KOMAoptions{parskip=half}}
\makeatother
\usepackage{xcolor}
\usepackage[margin=1in]{geometry}
\usepackage{graphicx}
\makeatletter
\def\maxwidth{\ifdim\Gin@nat@width>\linewidth\linewidth\else\Gin@nat@width\fi}
\def\maxheight{\ifdim\Gin@nat@height>\textheight\textheight\else\Gin@nat@height\fi}
\makeatother
% Scale images if necessary, so that they will not overflow the page
% margins by default, and it is still possible to overwrite the defaults
% using explicit options in \includegraphics[width, height, ...]{}
\setkeys{Gin}{width=\maxwidth,height=\maxheight,keepaspectratio}
% Set default figure placement to htbp
\makeatletter
\def\fps@figure{htbp}
\makeatother
\setlength{\emergencystretch}{3em} % prevent overfull lines
\providecommand{\tightlist}{%
  \setlength{\itemsep}{0pt}\setlength{\parskip}{0pt}}
\setcounter{secnumdepth}{-\maxdimen} % remove section numbering
\ifLuaTeX
  \usepackage{selnolig}  % disable illegal ligatures
\fi
\IfFileExists{bookmark.sty}{\usepackage{bookmark}}{\usepackage{hyperref}}
\IfFileExists{xurl.sty}{\usepackage{xurl}}{} % add URL line breaks if available
\urlstyle{same}
\hypersetup{
  pdftitle={Spring 2023 S\&DS 220 Project Description},
  colorlinks=true,
  linkcolor={Maroon},
  filecolor={Maroon},
  citecolor={Blue},
  urlcolor={blue},
  pdfcreator={LaTeX via pandoc}}

\title{Spring 2023 S\&DS 220 Project Description}
\author{}
\date{\vspace{-2.5em}}

\begin{document}
\maketitle

{
\hypersetup{linkcolor=}
\setcounter{tocdepth}{2}
\tableofcontents
}
\hypertarget{overview}{%
\section{Overview}\label{overview}}

This project will be an opportunity to apply the skills you've learned
in S\&DS 220 to a statistics or data analysis topic that is of interest
to you. In particular this project has the following objectives:

\begin{enumerate}
\def\labelenumi{\arabic{enumi}.}
\tightlist
\item
  Formulating a well-defined problem/question that will be investigated
  using data science skills and statistical tools.
\item
  Acquiring, cleaning, and organizing the data.
\item
  Data exploration and visualization.
\item
  Regression/hypothesis testing/inferential statistics
\item
  Interpretation of results
\item
  Communication. This includes:

  \begin{enumerate}
  \def\labelenumii{\alph{enumii}.}
  \tightlist
  \item
    having a well-organized professional looking report (generated from
    an R markdown document)
  \item
    clearly communicating the entire process and the interpretations of
    your results.
  \end{enumerate}
\end{enumerate}

\hypertarget{data-sources}{%
\section{Data sources}\label{data-sources}}

You will need (at least) one data set for your project. Data sets should
typically include several variables in columns (3 or more). If you are
doing regression or classification, there should be at least one outcome
column (numerical or categorical). The number of rows should at least be
in the hundreds.

There are many sources for data. Here are a few to help you get started
(though you are not restricted to these).

\begin{itemize}
\tightlist
\item
  \href{https://www.kaggle.com/datasets}{Kaggle datasets}
\item
  \href{https://cran.r-project.org/web/packages/Stat2Data/Stat2Data.pdf}{R
  Stat2Data package}
\item
  \href{https://stat.ethz.ch/R-manual/R-devel/library/datasets/html/00Index.html}{R
  datasets package}
\item
  \href{https://archive.ics.uci.edu/datasets}{UCI Machine Learning
  Repository}
\end{itemize}

Feel free to meet with the instructor or one of the TAs if you need help
obtaining data.

\hypertarget{doing-the-work}{%
\section{Doing the work}\label{doing-the-work}}

Each step in the data science pipeline is part of an iterative process.
You may start with a question in mind, and upon exploring data, decide
that you should be answering a different question. Often times the model
you build is insufficient (assumptions not met, etc\ldots), and so you
need to go back and refine the model. The following diagram highlights
some of the steps in this process.

\begin{figure}
\centering
\includegraphics{https://rviews.rstudio.com/post/2019-06-14-a-gentle-intro-to-tidymodels_files/figure-html/ds.png}
\caption{The data science modeling process, from:
\url{https://rviews.rstudio.com/2019/06/19/a-gentle-intro-to-tidymodels/}}
\end{figure}

A project template and further guidelines for your project write ups are
in the \emph{Project} folder on Canvas. Below is a rough outline, but
what goes in the report can be topic dependent.

\begin{itemize}
\tightlist
\item
  Introduction
\item
  Data exploration, visualization
\item
  Regression modeling, Analysis
\item
  Visualization and interpretation of the results
\item
  Conclusions and recommendations
\end{itemize}

Your report should be at least 5 pages, and no more than 10. It should
be concise, to-the-point, and polished. These sections don't have to be
long, only long enough to communicate what you need to communicate.

\hypertarget{project-submission-and-deadline}{%
\section{Project submission and
deadline}\label{project-submission-and-deadline}}

You will be responsible for submitting a short report in R Markdown with
your work.

\begin{itemize}
\tightlist
\item
  Please submit your \texttt{Report.pdf} on Gradescope as you would for
  another assignment.
\item
  The due date for the report is \textbf{Friday April 26th, 11:59PM}
\end{itemize}

\end{document}
