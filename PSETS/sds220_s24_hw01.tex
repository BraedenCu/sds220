% Options for packages loaded elsewhere
\PassOptionsToPackage{unicode}{hyperref}
\PassOptionsToPackage{hyphens}{url}
\PassOptionsToPackage{dvipsnames,svgnames,x11names}{xcolor}
%
\documentclass[
]{article}
\usepackage{amsmath,amssymb}
\usepackage{iftex}
\ifPDFTeX
  \usepackage[T1]{fontenc}
  \usepackage[utf8]{inputenc}
  \usepackage{textcomp} % provide euro and other symbols
\else % if luatex or xetex
  \usepackage{unicode-math} % this also loads fontspec
  \defaultfontfeatures{Scale=MatchLowercase}
  \defaultfontfeatures[\rmfamily]{Ligatures=TeX,Scale=1}
\fi
\usepackage{lmodern}
\ifPDFTeX\else
  % xetex/luatex font selection
\fi
% Use upquote if available, for straight quotes in verbatim environments
\IfFileExists{upquote.sty}{\usepackage{upquote}}{}
\IfFileExists{microtype.sty}{% use microtype if available
  \usepackage[]{microtype}
  \UseMicrotypeSet[protrusion]{basicmath} % disable protrusion for tt fonts
}{}
\makeatletter
\@ifundefined{KOMAClassName}{% if non-KOMA class
  \IfFileExists{parskip.sty}{%
    \usepackage{parskip}
  }{% else
    \setlength{\parindent}{0pt}
    \setlength{\parskip}{6pt plus 2pt minus 1pt}}
}{% if KOMA class
  \KOMAoptions{parskip=half}}
\makeatother
\usepackage{xcolor}
\usepackage[margin=1in]{geometry}
\usepackage{color}
\usepackage{fancyvrb}
\newcommand{\VerbBar}{|}
\newcommand{\VERB}{\Verb[commandchars=\\\{\}]}
\DefineVerbatimEnvironment{Highlighting}{Verbatim}{commandchars=\\\{\}}
% Add ',fontsize=\small' for more characters per line
\usepackage{framed}
\definecolor{shadecolor}{RGB}{248,248,248}
\newenvironment{Shaded}{\begin{snugshade}}{\end{snugshade}}
\newcommand{\AlertTok}[1]{\textcolor[rgb]{0.94,0.16,0.16}{#1}}
\newcommand{\AnnotationTok}[1]{\textcolor[rgb]{0.56,0.35,0.01}{\textbf{\textit{#1}}}}
\newcommand{\AttributeTok}[1]{\textcolor[rgb]{0.13,0.29,0.53}{#1}}
\newcommand{\BaseNTok}[1]{\textcolor[rgb]{0.00,0.00,0.81}{#1}}
\newcommand{\BuiltInTok}[1]{#1}
\newcommand{\CharTok}[1]{\textcolor[rgb]{0.31,0.60,0.02}{#1}}
\newcommand{\CommentTok}[1]{\textcolor[rgb]{0.56,0.35,0.01}{\textit{#1}}}
\newcommand{\CommentVarTok}[1]{\textcolor[rgb]{0.56,0.35,0.01}{\textbf{\textit{#1}}}}
\newcommand{\ConstantTok}[1]{\textcolor[rgb]{0.56,0.35,0.01}{#1}}
\newcommand{\ControlFlowTok}[1]{\textcolor[rgb]{0.13,0.29,0.53}{\textbf{#1}}}
\newcommand{\DataTypeTok}[1]{\textcolor[rgb]{0.13,0.29,0.53}{#1}}
\newcommand{\DecValTok}[1]{\textcolor[rgb]{0.00,0.00,0.81}{#1}}
\newcommand{\DocumentationTok}[1]{\textcolor[rgb]{0.56,0.35,0.01}{\textbf{\textit{#1}}}}
\newcommand{\ErrorTok}[1]{\textcolor[rgb]{0.64,0.00,0.00}{\textbf{#1}}}
\newcommand{\ExtensionTok}[1]{#1}
\newcommand{\FloatTok}[1]{\textcolor[rgb]{0.00,0.00,0.81}{#1}}
\newcommand{\FunctionTok}[1]{\textcolor[rgb]{0.13,0.29,0.53}{\textbf{#1}}}
\newcommand{\ImportTok}[1]{#1}
\newcommand{\InformationTok}[1]{\textcolor[rgb]{0.56,0.35,0.01}{\textbf{\textit{#1}}}}
\newcommand{\KeywordTok}[1]{\textcolor[rgb]{0.13,0.29,0.53}{\textbf{#1}}}
\newcommand{\NormalTok}[1]{#1}
\newcommand{\OperatorTok}[1]{\textcolor[rgb]{0.81,0.36,0.00}{\textbf{#1}}}
\newcommand{\OtherTok}[1]{\textcolor[rgb]{0.56,0.35,0.01}{#1}}
\newcommand{\PreprocessorTok}[1]{\textcolor[rgb]{0.56,0.35,0.01}{\textit{#1}}}
\newcommand{\RegionMarkerTok}[1]{#1}
\newcommand{\SpecialCharTok}[1]{\textcolor[rgb]{0.81,0.36,0.00}{\textbf{#1}}}
\newcommand{\SpecialStringTok}[1]{\textcolor[rgb]{0.31,0.60,0.02}{#1}}
\newcommand{\StringTok}[1]{\textcolor[rgb]{0.31,0.60,0.02}{#1}}
\newcommand{\VariableTok}[1]{\textcolor[rgb]{0.00,0.00,0.00}{#1}}
\newcommand{\VerbatimStringTok}[1]{\textcolor[rgb]{0.31,0.60,0.02}{#1}}
\newcommand{\WarningTok}[1]{\textcolor[rgb]{0.56,0.35,0.01}{\textbf{\textit{#1}}}}
\usepackage{graphicx}
\makeatletter
\def\maxwidth{\ifdim\Gin@nat@width>\linewidth\linewidth\else\Gin@nat@width\fi}
\def\maxheight{\ifdim\Gin@nat@height>\textheight\textheight\else\Gin@nat@height\fi}
\makeatother
% Scale images if necessary, so that they will not overflow the page
% margins by default, and it is still possible to overwrite the defaults
% using explicit options in \includegraphics[width, height, ...]{}
\setkeys{Gin}{width=\maxwidth,height=\maxheight,keepaspectratio}
% Set default figure placement to htbp
\makeatletter
\def\fps@figure{htbp}
\makeatother
\setlength{\emergencystretch}{3em} % prevent overfull lines
\providecommand{\tightlist}{%
  \setlength{\itemsep}{0pt}\setlength{\parskip}{0pt}}
\setcounter{secnumdepth}{-\maxdimen} % remove section numbering
\ifLuaTeX
  \usepackage{selnolig}  % disable illegal ligatures
\fi
\IfFileExists{bookmark.sty}{\usepackage{bookmark}}{\usepackage{hyperref}}
\IfFileExists{xurl.sty}{\usepackage{xurl}}{} % add URL line breaks if available
\urlstyle{same}
\hypersetup{
  pdftitle={S\&DS 220: Homework 1},
  pdfauthor={Braeden Cullen},
  colorlinks=true,
  linkcolor={Maroon},
  filecolor={Maroon},
  citecolor={Blue},
  urlcolor={blue},
  pdfcreator={LaTeX via pandoc}}

\title{S\&DS 220: Homework 1}
\author{Braeden Cullen}
\date{2024-01-19}

\begin{document}
\maketitle

\hypertarget{instructions}{%
\subsection{Instructions}\label{instructions}}

\begin{itemize}
\tightlist
\item
  This assignment is \textbf{due Friday January 19th, 11:59
  pm}\footnote{There is a 48 hour grace period, so you have until the
    end of Sunday.} in Gradescope. Download the R Markdown version of
  this document from Canvas (called ``sds220\_s24\_hw01.rmd'').
\item
  Be sure to put you name in the ``author'' section of the YAML header
  above.
\item
  This homework assignment consists of two parts.

  \begin{enumerate}
  \def\labelenumi{\arabic{enumi}.}
  \tightlist
  \item
    In Part I, you will work through R tutorials in the \texttt{swirl}
    package at your own pace. This will not be turned in, and you do not
    need to complete it before doing Part II.
  \item
    In Part II, you will be working the \texttt{nhtemp} data set, which
    is built into R. There are 5 questions, each worth 20 points. For
    each question, your solution will consist of writing R code (in a
    code chunk) or writing an explanation (outside of a code chunk after
    the prompt).
  \end{enumerate}
\item
  Submit your solutions in Gradescope as a pdf file generated from this
  document by clicking the \emph{Knit} button at the top of the window.
\end{itemize}

\newpage

\hypertarget{part-i-swirl-package-tutorials}{%
\subsection{\texorpdfstring{Part I: \texttt{swirl} package
tutorials}{Part I: swirl package tutorials}}\label{part-i-swirl-package-tutorials}}

First you need to install the \texttt{swirl} package. In the
\emph{console} window, type the command

\begin{quote}
\texttt{install.packages("swirl")}
\end{quote}

This will install the \texttt{swirl} package on your computer. R
packages are written by R users and shared with the R community through
a central repository. They contain extra tools and expand the
functionality base R. Some packages focus on specialized applications,
including finance, sports, astronomy, and gene sequencing (and much much
more).

Here is a
\href{https://cran.r-project.org/web/packages/available_packages_by_name.html}{list
of current R packages}. As you can see, the list is quite extensive. In
addition to this, some users create packages and share them from their
\href{https://github.com}{Github} account, rather than using the central
CRAN repository.

When you download a package, it goes into a folder on your computer
called a \emph{library}. However its contents are not available
immediately. Each time you open a new R session, you can make a package
available with the \texttt{library} function. Think of the folder that
contains R packages as a library (that's what R calls it after all!),
and the individual packages as books. The \texttt{library} function is
like going to the library and checking out a book (package).

Load the \texttt{swirl} package by typing the following command on the
command line in the console:

\begin{quote}
\texttt{library(swirl)}
\end{quote}

You will be greeted with a welcome message and prompted to type in the
command

\begin{quote}
\texttt{swirl()}
\end{quote}

to start the tutorial. Do so, and follow the prompts. When you get to
the list of courses, choose the first option:

\begin{quote}
1: R Programming: The basics of programming in R
\end{quote}

This course consists of 15 lessons, each of which takes about 10-15
minutes to complete. Work through these lessons at your own pace. You do
not need to complete them all before moving on to Part II of this
assignment. I recommend doing several lessons each day (or every couple
of days) over the next two weeks. You are welcome to do more and dive
into the other swirl courses! The more time you put into learning R, the
easier it will become and the more you'll be able to do.

\newpage

\hypertarget{part-ii-basic-r}{%
\subsection{Part II: Basic R}\label{part-ii-basic-r}}

\hypertarget{question-1}{%
\subsubsection{Question 1}\label{question-1}}

Write R code to create a numeric vector that repeats the sequence
\texttt{1,\ 1,\ 2,\ 3,\ 5} eight hundred times. Store the result as
\texttt{fib}.

\begin{Shaded}
\begin{Highlighting}[]
\CommentTok{\# your answer here}
\end{Highlighting}
\end{Shaded}

\newpage

\hypertarget{question-2}{%
\subsubsection{Question 2}\label{question-2}}

\texttt{nhtemp} is a dataset that is built into R. Create a histogram of
\texttt{nhtemp} using the \texttt{hist} function. Set the color of the
histogram to a color that you like (but something other than the
default).

\begin{Shaded}
\begin{Highlighting}[]
\CommentTok{\# your answer here}
\end{Highlighting}
\end{Shaded}

\newpage

\hypertarget{question-3}{%
\subsubsection{Question 3}\label{question-3}}

In the console window call the \texttt{help} function on
\texttt{nhtemp}. What is range of years for the data set?

\begin{quote}
\emph{Your answer here:} \_\_\_\_
\end{quote}

Call the function \texttt{str} on \texttt{nhtemp} to see what type of
object it is.

\begin{Shaded}
\begin{Highlighting}[]
\CommentTok{\# your answer here}
\end{Highlighting}
\end{Shaded}

\newpage

\hypertarget{question-4}{%
\subsubsection{Question 4}\label{question-4}}

What is the mean temperature in the \texttt{nhtemp} data?

\begin{Shaded}
\begin{Highlighting}[]
\CommentTok{\# your answer here}
\end{Highlighting}
\end{Shaded}

What is the standard deviation of the temperatures in the
\texttt{nhtemp} data?

\begin{Shaded}
\begin{Highlighting}[]
\CommentTok{\# your answer here}
\end{Highlighting}
\end{Shaded}

What is the maximum temperature in the \texttt{nhtemp} data?

\begin{Shaded}
\begin{Highlighting}[]
\CommentTok{\# your answer here}
\end{Highlighting}
\end{Shaded}

Which year had the maximum temperature in the \texttt{nhtemp} data? Do
not manually inspect the data for the largest value. Use the function
\texttt{which.max} and figure which year this corresponds to.

\begin{Shaded}
\begin{Highlighting}[]
\CommentTok{\# your answer here}
\end{Highlighting}
\end{Shaded}

\newpage

\hypertarget{question-5}{%
\subsubsection{Question 5}\label{question-5}}

\begin{itemize}
\tightlist
\item
  Call the \texttt{plot} function on \texttt{nhtemp}.
\item
  Change the label on the \emph{x}-axis to ``Year'' with the
  \texttt{xlab} argument.
\item
  Change the label on the \emph{y}-axis to ``Degrees Fahrenheit'' with
  the \texttt{ylab} argument.
\item
  Give the plot the title ``Temperatures in New Haven, CT from 1912 to
  1971'' with the \texttt{main} argument.
\end{itemize}

\begin{Shaded}
\begin{Highlighting}[]
\CommentTok{\# your answer here}
\end{Highlighting}
\end{Shaded}


\end{document}
